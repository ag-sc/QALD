

\subsection{DBpedia 3.10}
\label{sec:dbpedia}

DBpedia is a community effort to extract structured information from Wikipedia and to make this information available as RDF data. 
The RDF dataset provided for the challenge is the official DBpedia 3.10 dataset for English, including links, most importantly to YAGO\footnote{For detailed information on the YAGO class hierarchy, please see \url{http://www.mpi-inf.mpg.de/yago-naga/yago/}.} categories. This dataset comprises all files provided at: 
\begin{itemize}
\item \url{http://downloads.dbpedia.org/current/en/}
\item \url{http://downloads.dbpedia.org/current/links/}
\end{itemize}
In order to work with the dataset, you can either load it into your favorite triple store, or access it via a SPARQL endpoint. 
The official DBpedia SPARQL endpoint can be accessed at \url{http://dbpedia.org/sparql/}. 
%\begin{itemize} 
%\item[] \url{http://vtentacle.techfak.uni-bielefeld.de:443/sparql}
%\end{itemize}

Namespaces that are used in the provided training and test queries are the following ones:
\begin{itemize}
\item \emph{DBpedia:}
  \begin{itemize}
  \item[] \texttt{dbo: <http://dbpedia.org/ontology/>} 
  \item[] \texttt{dbp: <http://dbpedia.org/property/>} 
  \item[] \texttt{res: <http://dbpedia.org/resource/>} 
  \end{itemize}
\item \emph{RDF(S) and XSD:}
  \begin{itemize}
  \item[] \texttt{rdf:\ \ <http://www.w3.org/1999/02/22-rdf-syntax-ns\#>} 
  \item[] \texttt{rdfs: <http://www.w3.org/2000/01/rdf-schema\#>} 
  \item[] \texttt{xsd:\ \ <http://www.w3.org/2001/XMLSchema\#>}
  \end{itemize}
\item \emph{Others:}
  \begin{itemize}
  \item[] \texttt{yago: <http://dbpedia.org/class/yago/>} 
  \item[] \texttt{foaf: <http://xmlns.com/foaf/0.1/> } 
  \end{itemize}
\end{itemize}


\subsection{The questions}

In order to get acquainted with the dataset and possible questions, a set of training questions can be downloaded at the following location: 
\begin{itemize} 
\item \url{http://greententacle.techfak.uni-bielefeld.de/~cunger/qald/4/qald-5_train.xml} 
\end{itemize} 
This set comprises:
\begin{itemize}
\item 300 \emph{multilingual questions} 
\item 40 \emph{hybrid questions} 
\end{itemize}

\emph{Multilingual questions} are provided in seven different languages (English, German, Spanish, Italian, French, Dutch, and Romanian) 
and can be answered with respect to DBpedia 3.10. 
They are annotated with corresponding SPARQL queries and answers retrieved from the provided SPARQL endpoint. 

\emph{Hybrid questions} are provided in English and can be answered only by integrating structured data (RDF) and unstructured data 
(free text available in the DBpedia abstracts). The questions thus all require information from both RDF and free text. 
They are annotated with pseudo-queries that show which part is contained in the RDF data and which part has ti be retrieved  
from the free text abstracts. 


\subsubsection{Annotations}

Annotations are provided in the following XML format.
The overall document is enclosed by a tag that specifies an ID for the dataset indicating the domain 
and whether it is train or test (i.e. {\tt qald-4\_train} and {\tt qald-4\_test}).

\begin{lstlisting}[escapechar=!]
<dataset  id="qald-5_train">
<question id="1">   !\ldots! </question>
!\ldots!
<question id="340"> !\ldots! </question>
</dataset>
\end{lstlisting}

Each of the questions specifies an ID (don't worry if they are not ordered) together with the following attributes:
\begin{itemize}
\item {\tt answertype} gives the expected answer type, which can be one the following:
\begin{itemize}
\item {\tt resource}: One or many resources, for which the URI is provided.
\item {\tt string}: A string value such as {\tt Valentina Tereshkova}.
\item {\tt number}: A numerical value such as {\tt 47} or {\tt 1.8}.
\item {\tt date}: A date provided in the format {\tt YYYY-MM-DD}, e.g. {\tt 1983-11-02}. 
This format is also required when you submit results containing a date as answer.
\item {\tt boolean}: Either {\tt true} or {\tt false}.
\end{itemize}
% Answer of these types are required to be enclosed by the corresponding tag, i.e. {\tt <number>47</number>}, 
% {\tt <string>Valentina Tereshkova</string>} 
% and {\tt <boolean>true</boolean>}.
\item {\tt hybrid} specifies whether the question is a hybrid question, i.e. requires the use of both RDF and free text data 
\item {\tt aggregation} indicates whether any operations beyond triple pattern matching are required to answer the question (e.g., counting, filters, ordering, etc.).
\item {\tt onlydbo} reports whether the query relies solely on concepts from the DBpedia ontology. If the value is {\tt false}, the query might rely on 
the DBpedia property namespace (\texttt{http://dbpedia.org/property/}), FOAF or some YAGO category. 
\end{itemize}
Note that for hybrid questions, the attributes {\tt aggregation} and {\tt onlydbo} refer to the RDF part of the query only. 

Most importantly, each question specifies a question string and a corresponding query (both enclosed in {\tt <![CDATA[\ldots]]>} tags) 
as well as the correct answer(s). 
They slightly differ for multilingual and hybrid questions.


\subsubsection{Multilingual questions}

For multilingual questions, the question string is provided in seven languages: English, German, Spanish, Italian, French, Dutch, and Romanian, 
together with keywords in the same seven languages. The corresponding SPARQL query can be executed against the DBpedia endpoint in order to 
retrieve the specified answers. 
Here is an example:

\begin{lstlisting}[escapechar=|]
<question id="272" answertype="resource" 
                    aggregation="true" 
                    onlydbo="true"
                    hybrid="false">

<string lang="en">Which book has the most pages?</string>
<string lang="de">Welches Buch hat die meisten Seiten?</string>
<string lang="es">|\textquestiondown|Que libro tiene el mayor numero de paginas?</string>
<string lang="it">Quale libro ha il maggior numero di pagine?</string>
<string lang="fr">Quel livre a le plus de pages?</string>
<string lang="nl">Welk boek heeft de meeste pagina's?</string>
<string lang="ro">Ce carte are cele mai multe pagini?</string>

<keywords lang="en">book, the most pages</keywords>
<keywords lang="de">Buch, meisten Seiten</keywords>
<keywords lang="es">libro, mayor numero paginas</keywords>
<keywords lang="it">libro, maggior numero di pagine</keywords>
<keywords lang="fr">livre, le plus de pages</keywords>
<keywords lang="nl">boek, meeste pagina's</keywords>
<keywords lang="ro">carte, cele mai multe pagini</keywords>

<query>
PREFIX dbo: <http://dbpedia.org/ontology/>
PREFIX rdf: <http://www.w3.org/1999/02/22-rdf-syntax-ns#> 
SELECT DISTINCT ?uri
WHERE { 
        ?uri rdf:type dbo:Book . 
        ?uri dbo:numberOfPages ?n .
}
ORDER BY DESC(?n)
OFFSET 0 LIMIT 1
</query>

</question>
\end{lstlisting}

That is, all question strings and keywords have a language attribute (\texttt{lang}) with one of the following values: 
\texttt{en} (English), \texttt{de} (German), \texttt{es} (Spanish), \texttt{it} (Italian), \texttt{fr} (French), \texttt{nl} (Dutch), \texttt{ro} (Romanian).

As an additional challenge, a few of the training and test questions are out of scope, i.e. they cannot be answered with respect to DBpedia. 
The query is specified as {\tt OUT OF SCOPE} and the answer set is empty. Here is an example:

\begin{lstlisting}[escapechar=!]
<question id="99" answertype="resource" 
                  aggregation="false" 
                  onlydbo="false"
                  hybrid="false">

<string lang="en">
Give me all animal species that live in the Amazon rainforest.
</string>
!\ldots!

<query>
OUT OF SCOPE
</query>

<answers/>

</question>
\end{lstlisting}


\subsubsection{Hybrid questions}

The hybrid questions are partly based on the questions used in the INEX Linked Data track\footnote{\url{https://inex.mmci.uni-saarland.de/}} 2013. 
For a description of the track and the corresponding data, 
see \url{http://www.clef-initiative.eu/documents/71612/2b349f08-de37-41a9-bb62-40c91f1daa0b}.

For the hybrid questions, not only the RDF triples comprised by DBpedia are relevant, but also the English abstracts. 
They are related to a resource by means of the property \texttt{abstract}, e.g. as follows:

\begin{itemize}
\item[] \texttt{res:Australian\_Shelduck dbo:abstract \\[.1cm]
    `The Australian Shelduck, Tadorna tadornoides, is a shelduck, \\ 
     a group of large goose-like birds which are part of the bird \\ 
     family Anatidae. The genus name Tadorna comes from Celtic \\ 
     roots and means "pied waterfowl". They are protected under \\ 
     the National Parks and Wildlife Act, 1974.'@en .}
\end{itemize}

The questions are annotated with answers as well as a pseudo query that indicates which information from the RDF and which information 
from the free text abstracts have to be combined in order to find the answer(s). 
The pseudo query is like an RDF query but can contain free text as subject, property, or object of a triple. 
This free text is marked as \texttt{text:"\ldots"}. 
Here is an example: 

\begin{lstlisting}
<question id="335" answertype="resource" 
                   aggregation="false" 
                   onlydbo="true" 
                   hybrid="true">

<string lang="en">
Who is the front man of the band that wrote Coffee & TV?
</string>

<pseudoquery>
PREFIX dbo: <http://dbpedia.org/ontology/>
SELECT DISTINCT ?uri 
WHERE {
        <http://dbpedia.org/page/Coffee_&_TV> dbo:musicalArtist ?x .
        ?x dbo:bandMember ?uri .
        ?uri text:"is" text:"frontman" .
}      
</pseudoquery>
<answers>  
<answer>http://dbpedia.org/resource/Damon_Albarn</answer>
</answers>
</question>
\end{lstlisting}

The pseudo query contains three triples---two RDF triples and the third containing free text as property and object. 
The way to answer the question is to first retrieve the band members of the musical artist associated with the song Coffee \& TV 
using the triples
\begin{itemize}
\item[] \texttt{<http://dbpedia.org/page/Coffee\_\&\_TV> dbo:musicalArtist ?x .} \\
        \texttt{?x dbo:bandMember ?uri .} 
\end{itemize}
and then check the abstract of the returned URIs for the information whether they are the frontman of the band.
In this case, the abstract of Damon Albarn contains the following sentence: 
\begin{itemize}
\item[] \texttt{He is best known for being the frontman of the Britpop/alternative rock band Blur [\ldots]}
\end{itemize}

All queries are designed in a way that they require both RDF data and free text to be answered. 
Note that the free text given in the pseudo query corresponds to the natural language question, 
not the way the relevant information in the abstract is provided. Retrieving the relevant information  
from the abstracts thus usually requires some kind of textual entailment.
Also note that the pseudo queries cannot be evaluated against the SPARQL endpoint. 
